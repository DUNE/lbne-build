\documentclass[usletter]{article}

\pdfoutput=1            % must be in first 5 lines so arXiv finds it





\begin{document}
\section{Overview}

In this paper, we will address the following:
\begin{itemize}

	\item The FNAL `art/LArSoft' Software Stack
	\item The `UPS' Environment Management System
	\item The `cetbuildtools' CMake Add-ons
	\item New components
	\item The `worch' build automation tool
	\item Current Status
	\item Plan
	\item Interaction with other efforts
	\item Work Schedule

\end{itemize}

\section{Introduction}
LBNE must be able to install its required software from source code on
all major collaboration platforms (0). Whilst binary deployment systems such as
`cvmfs` or `rpm/deb` are useful, a simple, portable and automated from-source build mechanism must
be provided in order to easily create these binary packages and to ease
code development tasks. The single, largest hindrance in
satisfying this requirement that has been encountered by LBNE is the design and implementation of the low-level
CMake-based build system of the `LArSoft` and `art` packages.  These packages provide
the major components for the detector simulation portion of the LBNE software
stack and thus this hindrance has had a large impact on the progress of LBNE.

The main element of the `LArSoft/art` build system that is problematic
is the tight entanglement it has with the high-level `UPS` end-user
environment/package management system. Rather than having a configuration system
give configuration information to a build system, as the case in all industry
standard systems such as `rpm`/`deb`/`ports`/`homebrew`, the `LArSoft/art`
build system \textit{takes} configuration from `UPS`. This inversion of the usual
configuration-build hierarchy makes it impossible to build/run
the O(2000/10) C++11 sources/applications of `LArSoft/art`
without replicating the entire `UPS'-based software stack, down to the compiler
level. Though possible, this replication is difficult due to the installation
process being driven by a highly complex system of undocumented scripts, both
generated and hand coded in various languages (shell, Perl and Python).
It has been found that these factors essentially prevent porting `LArSoft/art'
to different systems, even those that provide compatible C++11/14 compilers
and the required third party software ``out of the box".

The heart of the problem, this inverted-dependency between the CMake build
and `UPS' configuration, was pointed out to the `art' development team in
April 2013, and in subsequent meetings both informal and official. The `art' team disagrees that this is a problem.
So we were left with a technical impasse, and we agreed to disagree on this point and all 
that follows from it.  However a plan was put forward in which LBNE would develop an alternative solution that did not suffer from this 
problem, propose it, and then the `art' team would consider it for adoption.

It should also be noted that LBNE's criticisms are not unique.   Other potential adopters of `art' (including
the CAPTAIN and SuperNEMO experiments) have rejected it due to the complexity
of the installation procedure despite its apparently small source footprint (more on this below).

The strategy of the solution is in two parts: The first part is to
decouple the `art/LArSoft' CMake scripts from `UPS' by rewriting these
using pure CMake functionality, hence increasing the portability and
usability of `LArSoft/art'. The second part is to remove the
configuration management logic and data that resided in the
UPS-entanglement and move it into a higher-level layer in the form of a
Worch configuration. This strategy has already been proven to work in an
initial conversion of the `art' packages and subsequent application to `LArSoft' packages.
The rest of this document describes more about the current status of this effort, a plan for
carrying forward this strategy and a rough time-line.

\section{The FNAL `art/LArSoft' Software Stack}
\subsection{Source and Dependency Footprints of `art/LArSoft'}

Both packages contain small amounts of C++11/14 source code (including all unit testing code):
\begin{itemize}

	\item FNAL foundation libraries contain in total O(200/200) C++ header/source files
	\item `Art' contains O(400/400) C++ header/source files
	\item `LArSoft' packages contain in total O(500/600) C++ header/source files

\end{itemize}


These in turn use the following standard and widely available third party packages:
\begin{itemize}

	\item Boost
	\item SQLite3
	\item ROOT
	\item CLHEP
	\item TBB

\end{itemize}
and only in LArSoft:

\begin{itemize}

	\item Geant4
	\item GENIE

\end{itemize}

These source/dependency footprints should be contrasted with core HEP
packages such as:

\begin{itemize}

	\item ROOT O(10000) sources, $>$ 10 external/optional dependencies
	\item Geant4 O(7000) sources, O(10) external/optional dependencies

\end{itemize}

These numbers are given to demonstrate that `art/LArSoft` are very simple and
lightweight packages by modern standards. It should also be noted that
neither `art` nor `LArSoft' have a large technical footprint. That is,
they follow the C++11/14 standard, and thus should not be tied to specific
architectures nor compilers.
These features should make `art/LArSoft' easy to build and install on any
system providing a C++11/14 compliant compiler plus the requisite packages.
LBNE's experience has been that this is not the case due to the coupling
of the build system to the `UPS' configuration management system, yet there
is no technical reason for this coupling to exist.
Neither ROOT or Geant4 require a specific configuration management
implementation to locate their required packages, making them highly portable
and easy to install despite their significantly larger source/dependency
footprint.

This is not to say that a configuration management system is not required
to help in integrating and managing an overall software stack.
Rather, the build systems of the packages comprising that stack should not
depend on a configuration management system being present, nor that it has a
specific implementation. This follows the basic software engineering
principle of separation of concerns.

\subsection{The `UPS' Environment Management System}

The `UPS' system provides software as ``products" which are
 collections of files with binaries for each
OS/architecture/compiler/optimization level/debug/profiling combination installed
in different directories. A product also contains table file(s)
with metadata on the product and which other products it depends on.
Configuration of packages for use by a user is through a series of
environment variables, not only the UNIX `PATH` variables but also many
undocumented per-package level variables. Though limited documentation
exists for `UPS`, it is out of date and poorly broken down into sections
for beginner and experienced users.

Although `UPS' is not ideal (it is noted that Fermilab's own pattern of use
`UPS' gives evidence of this) its use as the configuration management system
is in itself not a blocker for LBNE. Rather, it is the way that a hard
reliance on `UPS` has been built into the tools used to build `art',
`LArSoft` and their client packages. Clients of `LArSoft/art', not only
LBNE, are therefore unable to even *build* these packages without a full
`UPS` system replicated and configured locally, *even if the local system
provides all required packages already*.

It is noted that whilst `UPS` products are available through a `cvmfs'
repository, the intent and purpose of `cvmfs` is as a *deployment* system,
*not* a build system nor package manager. It is highly
useful for distributing software efficiently to a limited set of platforms,
but it provides no utility for building nor packaging that software easily
and cleanly or those or any other platform (and nor should it).

\subsection{The `cetbuildtools' CMake Add-ons}

`LArSoft/art' use the CMake build tool to configure, build and install
their runtime/development products. Many large software projects such as
ROOT, Geant4, LLVM and KDE (among others) have adopted CMake due to its
ease of use and portability, to quote from CMake's website:

\begin{quote}
\textit{CMake is a family of tools designed to build, test and package software.
 CMake is used to control the software compilation process using simple
 platform and compiler independent configuration files. CMake generates
 native makefiles and workspaces that can be used in the compiler
 environment of your choice.}
\end{quote}

To build a package, such as ROOT, using CMake to generate Makefiles, all one
does is
\begin{verbatim}
    $ cmake <args> /path/to/sourcedir
    $ make -j4
    $ make install
 \end{verbatim}

Custom options and configuration information, e.g. paths to needed packages,
can be passed through the command line arguments `$<$args$>$'. The `art/LArSoft'
packages should be similarly easy to build/install, but they are not for one
key reason: they require use of FNAL's `cetbuildtools' add-ons for CMake.
As an additional build-time dependency, `cetbuildtools' breaks the build
and use of `art/LArSoft' because of its design and implementation:
\begin{itemize}

	\item `cetbuildtools` is tightly coupled to FNAL's `UPS' configuration
  management system for finding things like GCC, Boost etc.
  
	\item This coupling and reliance on `UPS` is such that a user trying to build a
  package using `cetbuildtools` has no way to
  make it use system or any other non-`UPS' installs of the required packages,
  even if these meet all version requirements.
  
	\item `cetbuildtools' is highly specialized on using GCC as the compiler, and
  subsequently code in `art' has been found to contain GCCisms and code
  non-compliant with the C++11/14 standard.
  
	\item If a package \textbf{A} uses `cetbuildtools`, then a package \textbf{B} which uses \textbf{A}
  will be required to also use `cetbuildtools' (and thus `UPS'). This makes
  decoupling any client of `LArSoft/art' from `UPS' and `cetbuildtools' via
  a build-time ``firewall'' very difficult to implement.
  
	\item Most functionality in `cetbuildtools` demonstrates a fundamental
  lack of understanding of CMake and its capabilities/limitations (including
  package finding, import/export targets, target properties and globbing).
  
	\item Much of the `cetbuildtools` functionality is in the form of undocumented,
  heavyweight wrappers around core CMake functions, making the tool \textit{more
  difficult} to use. These wrappers also enforce source and binary layout
  conventions on the user which are of little benefit for either development
  or runtime use cases.
\end{itemize}


In short, `cetbuildtools' fails to implement a portable and easy to use
build interface.  Any project using `cetbuildtools' is
unbuildable without an entire local replication of a `UPS` stack. This is
an inversion of the usual hierarchy used in industry standard build systems,
e.g. a Makefile, sitting under a configuration/packaging system, e.g. RPM.

\section{New Components}
\subsection{The `worch` build automation tool}

The purification process leaves us with a package-level build system
which is no longer entangled with a build configuration system, as is
best practice.  Instead a higher level build configuration and
automation system based on `worch' (5)  is utilized.


Worch is a general purpose task automation tool with a focus on
developing a cohesive suite-wide build system.  Unlike the UPS/CET
build system it can be used to orchestrate the production of all of
LBNE software instead of just being limited to *art* based packages.


The Fermilab build system does not provide a concise method to define
a release and instead spreads configuration management across many
scripts, generated and hand written.  Worch allows for precise
description of the build configuration information in a single text
file or factored out, where it is convenient, into a few
domain-specific ones.  These simple text files completely define what
will be part of the release.  This includes which packages, their
versions and file system layout of the results.  This configuration
mechanism naturally lends itself to suite-wide release management
mechanisms based simply on keeping the configuration files in their
own code repository.


Unlike the Fermilab build system, worch does not place any
requirements on the packages it builds.  Because it does not assume
any policy but rather allows for policy to be expressed in the
configuration, it can produce a variety of build products from the
same source and at the same time.  For example, it can be used to
build UPS tarballs, Environment Modules, RPM and Debian packages or
others.


Also in contrast to the Fermilab build system which is a  collection
of intermixed scripts and configuration information, worch follows a
design that can be easily extended and in a way where extensions may
be shared by different projects.  Worch makes better use of build
hardware which is important for large software suites.  Besides being
automated if allows for inter-package parallelism.  Tasks run as
parallel as possible limited only by their inter-dependencies and the
available hardware.


Worch is a general purpose tool that will make managing the production
of releases of complex suites of software for multiple, partly related
experiments far easier than how things are currently done by Fermilab.
Worch is also simple enough that even groups with relatively modest
software stacks can benefit from the automation and configuration
management that it provides..


\subsection{Current Status}
The current status of the ``purification" of the low-level CMake build
system is described. Here the `art' packages are `cpp0x', `cetlib',
`fhicl-cpp' and `messagefacility' and `art' itself, and `LArSoft' refers to
its current count of ten packages.
\begin{itemize}

	\item An LBNE GitHub organization has been established.
	\item The `art' repositories are forked into this organization in a way
    that "upstream" commits pushed to Fermilab Redmine repositories
    continue to be tracked. Git's decentralized nature also means that
    commits on the GitHub repositories can be pulled into the Redmine
    repositories in the future.
	\item A new `FNALCore` package (2)  is developed that aggregates the `art`
    packages (except the `art` package itself) as well as holds their
    purified CMake files. An aggregation has been performed due to the
    heavy interdependencies between these packages and their intent as
    a foundation library for `art`.
	\item Purified CMake files are developed for `art' itself in the
    `fnal-art` repository (3).
	\item The `lbne-build` repository (4)  was created in the LBNE GitHub
    organization. It houses a Worch (5) configuration and tools to build
    all the 3rd-party external packages, `FNALCore` and `fnal-art` from
    source.
	\item The patterns applied to `art` packages have been pushed up the stack through most of `LArSoft`.
	\item Building these packages with Worch has been tested on at least
    Ubuntu (14.04) and Scientific Linux (6.4) and in a by-hand manner on
    Mac OS X.
\end{itemize}


\subsection{Plan}

The plan going forward is meant to satisfy these goals:
\begin{itemize}
	\item Push the commits of the purified CMake work into "upstream"
    repositories so that they no longer need to be held in separate
    tracking forks.
	\item Minimize disruption on the user base and provide an partly adiabatic
    change.
	\item Provide time for ongoing testing and improving of the purified CMake
    files while furthering and allowing the other goals.
\end{itemize}


The plan is in three major parts:
\begin{itemize}
	\item Continue to apply the CMake purification up through the LArSoft and the LBNE-specific
    `lbnecode` packages. In the same manner as with `fnal-art`, push
    commits to GitHub in forks which track their upstream repositories
    and in step add to `lbne-build` support to build each newly purified
    package. During this phase, Worch-related development is also needed
    in order to create UPS binary product "tarballs" from the build
    results and thus retain user-level status quo in the end.
	\item Change over from GitHub-based repositories to pushing commits to
    upstream repositories (in Fermilab Redmine/git). Do this by first purifying 
    `lbnecode` as above (and in GitHub) and then porting these changes into the
    `lbnecode' Redmine git repository with all changes placed behind a
    ``switch'' that defaults to the UPS-entangled build. Factor
    `lbne-build' to support building this "switched" pure-CMake
    `lbnecode` package against dependencies provided by UPS.
	\item With acceptance (and hopefully assistance) by the LArSoft group,
    continue porting the CMake purification, still kept switched off by
    default, to the LArSoft Redmine repositories and updating
    `lbne-build` to follow suit. Then, do likewise for the *art*
    packages. At some suitable point "flip the switch" so the entire
    stack is built in a pure-CMake manner with Worch.
\end{itemize}


\subsection{Interaction with other efforts}

Up until step three, this effort does not interfere with others. At step
three, buy-in by `LArSoft` and `art` developers and the Fermilab software
builders is required. However, before even making significant process on
step one it must be determined if Fermilab will accept the changes that
will be made in steps two and three. If not accepted, LBNE will revise
this plan  since it would be 
easier to remove the backward compatibility features, including removing UPS entirely, 
and use a git merging strategy to maintain the LBNE alternative build system.

\section{Work Schedule}

Due to the decision making and buy-in process described above, a
precise timeline for rollout of the plan cannot be given, however, we can provide  estimates
of effort needed for a specific, non-comprehensive set of items:
\begin{itemize}
	\item Complete purification of `LArSoft` (including `lbnecode`): 1 week FTE
	\item Complete worch feature to produce UPS tarballs from a built package:
  1 week FTE
	\item Port purification from GitHub repositories to Redmine ones
  (lbnecode, larsoft, art): 1 week FTE
\end{itemize}

Estimates are integrated coding time and do not take into account
necessary discussions, assume that *art* and `LArSoft` developers
agree to this approach and with not contingency to account for
unforeseen technical problems.  Also not included is defining and
enacting a campaign of testing and validation.

% [^0]: "LBNE Software and Computing Requirements", LBNE DocDB 8035


%[^1]: <https://github.com/LBNE> and links there for the individual  `fnal-*` forks.

%[^2]: <https://github.com/LBNE/FNALCore>

%[^3]: <https://github.com/LBNE/fnal-art>

%[^4]: <https://github.com/LBNE/lbne-build>

%[^5]: <https://github.com/brettviren/worch>

\end{document}



